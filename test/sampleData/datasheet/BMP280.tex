\documentclass[a4paper,12pt,oneside,pdflatex,italian,final,twocolumn]{article}

\usepackage[utf8]{inputenc}
\usepackage{parallel}
\usepackage{siunitx}
\usepackage{booktabs}
\usepackage{fancyhdr}

\usepackage[export]{adjustbox}
\usepackage[margin=0.5in]{geometry}
\addtolength{\topmargin}{0in}

\usepackage{libertine}
\renewcommand*\familydefault{\sfdefault}  %% Only if the base font of the document is to be sans serif
\usepackage[T1]{fontenc}

\title{ BMP280 }
\author{ Nick Felker }
\date{ 2019 }

\begin{document}

\pagestyle{fancy}

\lhead{ Nick Felker }
\chead { 2019 }
\rhead{ BMP280 v0.1.0 }


\onecolumn


\begin{figure}
\begin{minipage}{0.47\textwidth}

\section{Overview}
    Bosch Digital Pressure Sensor
    \begin{itemize}
        \item Device address 119
        \item Address type 7-bit
    \end{itemize}


\end{minipage}
\hfill

\end{figure}


\section{Register Description}
\begin{itemize}
\item Digital Pressure Compensation 1 - Compensation register used for Pascals conversion.
\item Digital Pressure Compensation 2 - Compensation register used for Pascals conversion.
\item Digital Pressure Compensation 3 - Compensation register used for Pascals conversion.
\item Digital Pressure Compensation 4 - Compensation register used for Pascals conversion.
\item Digital Pressure Compensation 5 - Compensation register used for Pascals conversion.
\item Digital Pressure Compensation 6 - Compensation register used for Pascals conversion.
\item Digital Pressure Compensation 7 - Compensation register used for Pascals conversion.
\item Digital Pressure Compensation 8 - Compensation register used for Pascals conversion.
\item Digital Pressure Compensation 9 - Compensation register used for Pascals conversion.
\item Digital Temperature Compensation 1 - Compensation register used for temperature measurement result.
\item Digital Temperature Compensation 2 - Compensation register used for temperature measurement result.
\item Digital Temperature Compensation 3 - Compensation register used for temperature measurement result.
\item Pressure LSB - Middle-significant bit of pressure measurement result.
\item Pressure MSB - Most-significant bit of pressure measurement result.
\item Pressure XLSB - Least-significant bit of pressure measurement result.
\item Temperature LSB - Middle-significant bit of temperature measurement result.
\item Temperature MSB - Most-significant bit of temperature measurement result.
\item Temperature XLSB - Least-significant bit of temperature measurement result.
\end{itemize}

\section{Technical specification}
\centering
\begin{tabular}{lcrr}
\toprule
 & Register Name & Register Address & Register Length & Example \\
\midrule
DigP1 & Digital Pressure Compensation 1 & 142 & 16
& 36477 
\\
DigP2 & Digital Pressure Compensation 2 & 144 & 16
& -10685 
\\
DigP3 & Digital Pressure Compensation 3 & 146 & 16
& 3024 
\\
DigP4 & Digital Pressure Compensation 4 & 148 & 16
& 2855 
\\
DigP5 & Digital Pressure Compensation 5 & 150 & 16
& 140 
\\
DigP6 & Digital Pressure Compensation 6 & 152 & 16
& -7 
\\
DigP7 & Digital Pressure Compensation 7 & 154 & 16
& 15500 
\\
DigP8 & Digital Pressure Compensation 8 & 156 & 16
& -14600 
\\
DigP9 & Digital Pressure Compensation 9 & 158 & 16
& 6000 
\\
DigT1 & Digital Temperature Compensation 1 & 136 & 16
& 27504 
\\
DigT2 & Digital Temperature Compensation 2 & 138 & 16
& 26435 
\\
DigT3 & Digital Temperature Compensation 3 & 140 & 16
& -1000 
\\
PressureLsb & Pressure LSB & 248 & 8
& 33400 
\\
PressureMsb & Pressure MSB & 247 & 8
& 15600 
\\
PressureXlsb & Pressure XLSB & 249 & 8
& 15920 
\\
TempLsb & Temperature LSB & 251 & 8
& 748 
\\
TempMsb & Temperature MSB & 250 & 8
& 124 
\\
TempXlsb & Temperature XLSB & 252 & 8
& 144 
\\
\bottomrule
\end{tabular}


\raggedright

\section{Functions}

\centering
\begin{tabular}{lc}
\toprule
  & Description \\
\midrule
pressure & Reads the atmospheric pressure as a raw value or in hPa. \\
temperature & Reads the temperature as a raw value or in Celsius. \\
\bottomrule
\end{tabular}


\raggedright
\subsection{Function pressure }
Reads the atmospheric pressure as a raw value or in hPa. \\

\centering
\begin{tabular}{lcr}
\toprule
  & Inputs & Return \\
\midrule
asHpa &
&
hpa
\\
asRaw &
&
output
\\
\bottomrule
\end{tabular}



\raggedright
\subsection{Function temperature }
Reads the temperature as a raw value or in Celsius. \\

\centering
\begin{tabular}{lcr}
\toprule
  & Inputs & Return \\
\midrule
asCelsius &
&
celsius
\\
asRaw &
&
output
\\
\bottomrule
\end{tabular}



\raggedright

\end{document}